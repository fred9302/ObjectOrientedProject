\chapter{Discussion}\label{ch:Discussion}


\section{GUI}
When a user of the system lay their eyes on the GUI, the first elements, they will see, are inputs for how many columns and rows should be in the grid that makes up the system as can be seen in \autoref{fig:GUI_screenshot_1}. Even though the user might be quite knowledgeable about networking, the will only be able to guess what this grid is. Therefore, an easy to implement solution could be to provide a simple description of the use case of the system. An even better solution could be to provide an image that changed along with the users input, and then provide some visualisation of the size of the grid such as axes to clarify even further that the user is changing the graph or grid that will make up the area where the devices in the network simulation can occupy.

\begin{figure}[H]
  \centering
  \includegraphics[width=\textwidth]{GUI_screenshot_1.pdf}
  \caption{A screenshot of the first elements that are displayed when a user opens the GUI.}
  \label{fig:GUI_screenshot_1}
\end{figure}

Then, after the simulation has finished running, average metrics of the results from the individual nodes are displayed printed as is shown in \autoref{fig:GUI_screenshot_2}. An improvement to this implementation might incorporate an option to view the results from each individual node by either having the results in a simple table or implementing some way to display the results by having the user click on a button with the name of the node (i.e. Node n results). The option of using an element that is already present would be to enable the network topology graph to function as buttons, where the user would be able to click on the number of a given node and then have the results displayed. An example of the network topolgy graph can be seen in 

\begin{figure}[H]
  \centering
  \includegraphics[width=0.5\textwidth]{GUI_screenshot_2.pdf}
  \caption{A screenshot of how the results of the simulation is outputted in GUI.}
  \label{fig:GUI_screenshot_2}
\end{figure}



\section{Network}
As has already been described in \autoref{sec:Class_diagram_Network} and in the rest of \autoref{sec:Class_diagram}, a better implementation of how the system currently is implemented would have tried to get as close to functional cohesion as possible in order clarify and improve which object does what. As the implementation currently is, a name that more precisely describes the functionality of the \code{simulation} object would be \code{simulation\_and\_networking}. Of course, it must be acknowledged that complete functional cohesion will not make sense in this project, where


\section{Device}


\subsection{Node}


\subsection{Gateway}


\section{Simulation}



































