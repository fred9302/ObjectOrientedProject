% Shortcuts
\usetikzlibrary{quotes,angles}

\newcommand\w{Wi-Fi\xspace}

\newcommand\rpi{Raspberry Pi\xspace}
\newcommand\rpis{Raspberry Pis\xspace}

\newcommand\ts{timestamp\xspace}
\newcommand\tss{timestamps\xspace}
\newcommand\Ts{Timestamp\xspace}
\newcommand\Tss{Timestamps\xspace}

\newcommand\tsing{timestamping\xspace}
\newcommand\Tsing{Timestamping\xspace}

\newcommand\cn{communication network\xspace}
\newcommand\Cn{Communication network\xspace}

\newcommand\cs{communication system\xspace}
\newcommand\Cs{Communication system\xspace}

\newcommand\ds{distributed system\xspace}
\newcommand\Ds{Distributed system\xspace}

\newcommand\dn{distributed network\xspace}
\newcommand\Dn{Distributed network\xspace}

\newcommand\ah{ad-hoc network\xspace}
\shortcut\ahs{ad-hoc networks}
\newcommand\AH{Ad-hoc network\xspace}
\shortcut\Ah{Ad-hoc network}

\newcommand{\marker}[1]{\colorbox{BurntOrange}{#1}}
\newcommand{\ctext}[3][RGB]{%
  \begingroup
  \definecolor{hlcolor}{#1}{#2}\sethlcolor{hlcolor}%
  \hl{#3}%
  \endgroup
}
% The \shortcut command is the same as \newcommand, it just adds \xspace itself instead of you having to do it
\shortcut\classifications{throughput, delay, and packet loss}
\shortcut\Classifications{Throughput, delay, and packet loss}
\shortcut\classificationsor{throughput, delay, or packet loss}
\shortcut\Classificationsor{Throughput, delay, or packet loss}
%--------------------------------------------

%\newcommand\shortcut[2]{\newcommand#1{#2\xspace}}

\newcommand{\code}[1]{\texttt{#1}}